\documentclass[12pt]{article}

\usepackage[T1]{fontenc}
\usepackage[pdftex]{graphicx}
\usepackage[hmargin=2cm,vmargin=3cm]{geometry}
\usepackage{float}

\title{TDDD17 lab 1}
\author{Gustav Ahlberg gusah849 \\ Claire Vacherot clava401}

\begin{document}
\maketitle

\newpage

\section*{Exercise 1}

When the user signs up for the website, he also has to download a specific app for his smart-phone and scan a QR-code that pairs with the phone to the newly created account to activate it. Then when the user wants to log in he has to supply his username to the website. He gets a response with a QR-code printed on the screen he has to scan with the app on his smart-phone. The application generates a one-time password based on the information in the QR-code. These information can only be interpreted by this specific application and not using other QR-code readers. The user then supplies the website with this one-time password to authenticate himself.

This is more secure than a ordinary password because it uses something that the user owns rather than what the user knows. To be able to get access to the account the attacker has to get physical access to the phone rather then just steal his password. That means that the attacker can't launch an attack on the user over the internet and the pool of possible attackers shrinks. The authenticated application can eventually be faked but this requires even more work to get to compromise it, and then after get access to the account. This can be avoided by authenticating the phone itself by logging its identity while accessing the website. The application can also be more secure if the user has to authenticate with an ordinary password before the QR-code can be scanned.


\begin{figure}[H]
%\includegraphics[width=\linewidth]{usecase.pdf}
\caption{To be able to login the user has to first type in the username, generate a one-time password using the application and then use the one-time password to log in.}
\label{fig:use-case}
\end{figure}


\begin{figure}[H]
%\includegraphics[width=\linewidth]{sequencediagram.pdf}
\caption{The sequence of actions to perform to authenticate with this method.}
\label{fig:sequence-diagram}
\end{figure}

\section*{Exercise 2}
\subsection*{2-1}
\subsubsection*{OpenID}
OpenID is used to authenticate a specific user by giving its identity, which is granted or not by the provider. The application has to know and verify using a certificate who you are to give the permission to access.

\subsubsection*{OAuth}
OAuth is used to authenticate by using a key to get access given by the provider which has identified the user/process. This means that the application you log into doesn't need to know who you are, just that you have the permission (key) to access it on user's behalf.

\subsection*{2-2}
\begin{enumerate}
\item OpenID delivers a certificate (the information needed to get access) whereas OAuth delivers a key (the means to access directly)
\item OAuth is about the user authorize the application to do things on the users behalf and OpenID authenticate the user so that the user can access the application.
\item OpenID only communicate with the user and OAuth allows the application the communicate with other applications that the user has authorized access to. 
\item With OpenID the user can login to multiple websites using the same account information. With OAuth the user can share resources between applications.
\item With OpenID each application sees the user as a unique user and he has to specifically login to all the websites that uses OpenID. With OAuth the user only have to authenticate once to get access to several applications. 
\end{enumerate}

\subsection*{2-3}
For both the application, being able to log in once extends the scope of the attacker (as you gain access to more than one account/application): For OpenID, getting the log in information for one means that you got them for everything. For OAuth being logged in to one application can mean you are logged in to a lot of them.

\section*{Exercise 3}


\end{document}
